\documentclass[11pt]{article}
\usepackage{pslatex}
\usepackage{rotating}
\usepackage{footnote}
\usepackage{multirow}
\usepackage{timing}

% commands for register format figures.

\newcommand{\instbit}[1]{\mbox{\scriptsize #1}}
\newcommand{\instbitrange}[2]{\instbit{#1} \hfill \instbit{#2}}

\setlength{\topmargin}{-0.25in}
\setlength{\headheight}{0.25in}
\setlength{\headsep}{0.25in}
\setlength{\textheight}{8.5in}

\setlength{\oddsidemargin}{0.0in}
\setlength{\evensidemargin}{0.0in}
\setlength{\textwidth}{6.5in}

\renewcommand{\textfraction}{0.05}
\renewcommand{\topfraction}{0.95}
\renewcommand{\bottomfraction}{0.95}

% Define \colw giving text width in a single column.
\newlength{\colw}
\setlength{\colw}{\textwidth}

\makeatother

\renewcommand{\textfraction}{0}
\renewcommand{\topfraction}{1}
\renewcommand{\bottomfraction}{1}
\sloppy

% atb0 specific commands

\pagestyle{myheadings}
%\pagestyle{empty}

% symbol footnotes
\long\def\symbolfootnote[#1]#2{\begingroup%
\def\thefootnote{\fnsymbol{footnote}}\footnote[#1]{#2}\endgroup}

\let\subsubsubsection\paragraph
\setcounter{secnumdepth}{5}
\setcounter{tocdepth}{5}


\begin{document}

\title{\Large \bf Tri-Mode Ethernet MAC v0.1}
\author{Jared Casper \\ FOSSH}
\maketitle

\newpage

\newcommand{\headertext}{Tri-Mode Ethernet MAC v0.1}
\markboth{\headertext}{\headertext}

%%\tableofcontents
%%\newpage

%%\listoffigures
%%\newpage

%% \listoftables
%% \newpage

\section{Introduction}

FOSSH's Tri-mode Ethernet MAC core was designed from the ground up to
be a compact, fast, no-frills core to faciliate streaming data from a
widget to a connected computer.  Because the core is meant to be small
and fast, many features that are not often used or would not be useful
for streaming data are left out.  For example, the core only supports
full-duplex operation in that it does not even look at the CRS
(carrier-sense) and COL (collision detection) signals from the PHY.
That said, it does have features not present in other cores to make it
easy to stream data.  For example, it will generate the ethernet and
udp headers for you so you can simple give it data and it will take
care of sending it for you.  

The core is in active development and this document describes version
0.1.

\subsection{Features}

Features that are implemented and functional:

\begin{itemize}

\item Interfaces with the PHY using GMII running 10/100/1000 Mb/s.
  Support for RGMII is planned.

\item Full-duplex support only.

\item Raw ethernet frame generation, allowing a pure data input stream.

\item Optional UDP/IP packet generation with automatic packetization and
  framentation of an input data stream.

\item Input and output fifos decoupled from the core logic, so the
  clock rate of application logic is not tied to the speed of the
  ethernet link (barring bandwidth issues).

\item Full core uses 288 Spartan 6 logic slices and 2 block RAMs.  For
  comparison, the smallest Spartan 6 has 600 slices and 12 block RAMs.
  The largest Spartan 6 has 23,038 slices and 268 block RAMs.

\end{itemize}

\noindent Features that are in development or planned:

\begin{itemize}


\item Multiple input/output fifo for streams to be sent to/from
  different addresses.

\item Optional configuration registers instead of configuration ports
  for use with a micro controller.

\item Automatic flow control via PAUSE frames.

\item RGMII interface.

\end{itemize}

\subsection{Obtaining}

Stable releases of the MAC core can be downloaded from
http://www.fossh.com/.  Read-only public access to the development
repository is available using git at the respository
git://jcasper.stanford.edu/fossh/repo.git.

\section{Interface}

\subsection{Parameters}
\begin{tabular}{lcp{4.0in}}
\hline
Name & Type & Description \\
\hline

\hline USR\_CLK\_PERIOD & Integer & Period, in ns, of usr\_clk.
Needed to to generate a sutable clock for the mii management port
(mdc) from usr\_clk. \\

\hline RX\_CRC\_CHECK & Boolean & If true, the CRC of incoming packets
will be checked for accuracy.  If false, no such check is performed. \\

\hline
\end{tabular}

\subsection{Clocks, etc.}
\begin{tabular}{p{0.85in}ccp{4.0in}}
\hline
Name & Width & Direction & Description \\
\hline

\hline usr\_clk & 1 & I & User clock used to clock data into and out
of the core through the user and management interfaces. \\

\hline clk\_125 & 1 & I & 125 MHz clock used to supply phy\_GTX\_CLK. \\

\hline reset\_n & 1 & I & Active-low reset.  Resets all flops and
buffers in the core. \\

\hline debug & * & O & Various signals used during debugging. \\
\hline
\end{tabular}

\subsection{User Interface}

All signals in the user interface are synchronous with usr\_clk.

\subsubsection{Configuration}

Instead of implementing internal registers to indicate the source and
destination MAC address, these values are simply ports into the core.
This allowed the implementation to be expanded to a wider variety of
future interfaces.  An optional wrapper is in development that will
add registers to the management interface
(Section~\ref{sec:management}) and remove these ports.

The following ports are part of the base MAC module and are always used:
\begin{tabular}{p{0.85in}ccp{4.0in}}
\hline
Name & Width & Direction & Description \\
\hline

\hline gmii & 1 & I & Indicate that the core should be communicate
with the phy in GMII mode (i.e. 1Gbit/s).  Otherwise MII mode
(10/100Mbit/s) will be used. Unfortunately there doesn't appear to be
a standard way to get the negotiated link speed from the PHY (why is
this not in the 802.3 standard is beyond me). \\

\hline promiscous & 1 & I & Indicate that the core should filter
incoming packages to those with destination of src\_mac\_addr. \\

\hline jumboframes & 1 & I & Indicates that jumbo frames are okay to
send. \\

\hline src\_mac\_addr & 48 & I & MAC address to use as the source of
transmitted ethernet frames and the filter for what received frames to
not discard when not in promiscous mode. \\

\hline dst\_mac\_addr & 48 & I & MAC address to use as the destination
of transmitted ethernet frames. \\

\hline ethertype & 16 & I & EtherType/Length field for outgoing packets. \\

\hline ifg & 10 & I & Number of cycles inserted between packets.  Should be $\geq$ 0x0D. \\

\hline
\end{tabular}

\subsubsection{Transmission}
\label{sec:tx-iface}

The interface used to send packets/frames is a standard FIFO write
interface with addition of an end-of-frame port ($tx\_eof$) that is
used to split incoming data into frames.

\vskip 12pt\noindent\begin{tabular}{p{0.85in}ccp{4.0in}}
\hline
Name & Width & Direction & Description \\
\hline

\hline tx\_data & 32 & I & Data to be sent to the phy for
transmission.  Data is queued to be sent when tx\_we is high and
tx\_stop is low.  Following ethernet standards, bytes are sent in big
endian order\\

\hline tx\_eof & 4 & I & Used to indicate which bytes in the word
being written (tx\_data) is the last of the frame.  Because data is
sent big endian, setting tx\_eof[0] to 1 indicates that the entire
32-bit should be sent (and is the last word in this packet). \\

\hline tx\_we & 1 & I & Write enable for transmission.  Queues
tx\_data to be sent to the phy when tx\_stop is low. \\

\hline tx\_stop & 1 & O & Indicates that the input buffers are full
and no more data can be queued for transmission. \\

\hline 
\end{tabular}\vskip 12pt

Figure~\ref{fig:timing-trans} illustrates normal transmission of a couple of
32-bit aligned packets, with the final word of the second packet being
temporarily stopped by a full input fifo.

\begin{figure}
\begin{timing}[2s]{0.6in}
\tin{1}{usr\_clk}  \til{1}{LLLLHHHHLLLLHHHHLL..LLHHHHLLLLHHHHLLLLHHHHLL..LLHHHHLLLLHHHHLLLLHHHH}
\tin{2}{tx\_data}  \til{2}{UUUUVVVVVVVVXVVVVV..VVXVVVVVVVXVVVVVVVXVVVVV..VVXVVVVVVVVVVVVVVVUUUU}
\tnote{1.85}{6}{data 1.1}\tnote{1.85}{15}{data 1.x}\tnote{1.85}{24}{data 1.n}
\tnote{1.85}{32}{data 2.1}\tnote{1.85}{41}{data 2.x}\tnote{1.85}{54}{data 2.n}
\tin{3}{tx\_eof[0]}\til{3}{LLLLLLLLLLLLLLLLLL..LLHHHHHHHHLLLLLLLLLLLLLL..LLHHHHHHHHHHHHHHHHLLLL}
\tin{4}{tx\_we}    \til{4}{LLLLHHHHHHHHHFFFFF..FHHHHHHHHHHHHHHHHHHFFFFF..FHHHHHHHHHHHHHHHHHLLLL}
\tin{5}{tx\_stop}  \til{5}{LLLLLLLLLLLLLLLLLL..LLLLLLLLLLLLLLLLLLLLLLLL..LLHHHHHHHHLLLLLLLLLLLL}
\sline{0.4}{4}{5.}
\sline{0.4}{12}{5.}
\sline{0.4}{22}{5.}
\sline{0.4}{30}{5.}
\sline{0.4}{38}{5.}
\sline{0.4}{48}{5.}
\sline{0.4}{56}{5.}
\sline{0.4}{64}{5.}
\end{timing}
\caption{Timing diagram of two back to back 32-bit aligned packet
  transmissions, with the final word of the second packet being
  interrupted by a full input fifo.}
\label{fig:timing-trans}
\end{figure}

If the input fifo runs out of data while the MAC is sending a packet,
an error condition will be raised and the packet will not be sent.
Operating at 1Gbps, the PHY will transmit approximately 100MB/s from
the input fifo.  Therefore, if data word is supplied every clock, the
user clock must be at least 25 MHz, an easy target for most FPGAs.
Alternatively, if the user clock is faster, data need not be supplied
on each clock.  For example, a 100 MHz user clock means that once data
is first written to the input fifo, the rest of the data can come at a
rate as slow as one word every 4 clocks.

It is important to note that bytes in tx\_data are sent in big endian
mode, following ethernet standards.  This means tx\_data[31:16] is
sent first, followed by tx\_data[15:8], and so on.  Care should be
taken if value smaller than 32 bits are written and their ordering.

\subsubsection{Reception}

The interface used to receive packets is a standard FIFO read
interface, with the addition of an end-of-frame port ($rx\_eof$).

\vskip 12pt\noindent\begin{tabular}{p{0.85in}ccp{4.0in}}
\hline
Name & Width & Direction & Description \\
\hline

\hline rx\_data & 32 & O & Data received by the MAC from the
phy. Valid when rx\_dv is high. \\

\hline rx\_eof & 4 & O & Used to indicate that rx\_data is the last
word in the received frame. Mapped to rx\_data just as tx\_eof maps to
tx\_data. \\

\hline rx\_dv & 1 & O & Indicates that rx\_data is valid data that was
received by the MAC. \\

\hline rx\_ack & 1 & I & Acknowledges the receipt of rx\_data and
causes the MAC to move to the next received word. \\

\hline
\end{tabular}\vskip 12pt

\noindent Figure~\ref{fig:timing-recp} illustrating the timing of
receiving a packet.  If the user logic does not empty the reception
fifo fast enough, it will fill up and the the MAC will drop incoming frames.  

\begin{figure}
\begin{timing}[2s]{0.6in}
\tin{1}{usr\_clk}  \til{1}{LLLLHHHHLLLLHHHHLL..LLHHHHLLLLHHHHLLLLHHHHLL..LLHHHHLLLLHHHHLLLLHHHH}
\tin{2}{rx\_data}  \til{2}{UUUUVVVVVVVVXVVVVV..VVXVVVVVVVXVVVVVVVXVVVVV..VVXVVVVVVVVVVVVVVVUUUU}
\tnote{1.85}{6}{data 1.1}\tnote{1.85}{15}{data 1.x}\tnote{1.85}{24}{data 1.n}
\tnote{1.85}{32}{data 2.1}\tnote{1.85}{41}{data 2.x}\tnote{1.85}{54}{data 2.n}
\tin{3}{rx\_eof[0]}\til{3}{LLLLLLLLLLLLLLLLLL..LLHHHHHHHHLLLLLLLLLLLLLL..LLHHHHHHHHHHHHHHHHLLLL}
\tin{4}{rx\_dv}    \til{4}{LLLLHHHHHHHHHFFFFF..FHHHHHHHHHHHHHHHHHHFFFFF..FHHHHHHHHHHHHHHHHHLLLL}
\tin{5}{rx\_ack}   \til{5}{LLLLLLLHHHHHHFFFFF..FHHHHHHHHHHHHHHHHHHFFFFF..FFLLLLLLLLHHHHHHHHLLLL}
\sline{0.4}{4}{5.}
\sline{0.4}{12}{5.}
\sline{0.4}{22}{5.}
\sline{0.4}{30}{5.}
\sline{0.4}{38}{5.}
\sline{0.4}{48}{5.}
\sline{0.4}{56}{5.}
\sline{0.4}{64}{5.}
\end{timing}
\caption{Timing diagram of receiving two back to back 32-bit aligned
  packets, with the final word of the second packet being interrupted
  by the user.}
\label{fig:timing-recp}
\end{figure}


\subsubsection{Management}
\label{sec:management}

The following ports are used to read and write the MII management
registers on the PHY.  If they are left unconnected, the MII
management module will be optimized away by the synthesis tools.

\vskip 12pt\noindent\begin{tabular}{p{0.85in}ccp{4.0in}}
\hline
Name & Width & Direction & Description \\
\hline

\hline miim\_phyad & 5  & I & Management address of the PHY to
communicate with. \\

\hline miim\_addr  & 5  & I & Address of PHY register to read/write
to. \\

\hline miim\_wdata & 16 & I & Data to write to the PHY register. \\

\hline miim\_we    & 1  & I & Write Enable, write miim\_wdata to the PHY
register miim\_addr. \\

\hline miim\_rdata & 16 & O & Data read from the PHY register. \\

\hline miim\_req   & 1  & I & Sends a read request to the PHY for
miim\_addr. \\

\hline miim\_ack   & 1  & O & Indicates that the requested operation
(read or write) has been performed.  If a read was requested,
miim\_rdata contains the data read from the PHY. \\

\hline
\end{tabular}

The management interface is a simple memory-style read/write
interface.  To read a management register, put the address on
\texttt{miim\_phyad} and \texttt{miim\_addr} and hold
\texttt{miim\_req} high until \texttt{miim\_ack} is raised.
\texttt{miim\_rdata} will hold the read data the cycle
\texttt{miim\_ack} is raised.  This is shown in
Figure~\ref{fig:timing-miim-rd}.

\begin{figure}
\begin{timing}[2s]{1in}
\tin{1}{usr\_clk}    \til{1}{LLLLHHHHLLLLHHHHLL....LLHHHHLLLLHHHHLLLLHHHHLL}
\tin{2}{miim\_phyad} \til{2}{UUUUVVVVVVVVVVVVVV....VVVVVVVVVVVVVVVVVVUUUUUU}
\tin{3}{miim\_addr}  \til{3}{UUUUVVVVVVVVVVVVVV....VVVVVVVVVVVVVVVVVVUUUUUU}
\tin{4}{miim\_req}   \til{4}{LLLLHHHHHHHHHHHHHH....HHHHHHHHHHHHHHHHHHLLLLLL}
\tin{5}{miim\_we}    \til{5}{LLLLLLLLLLLLLLLLLL....LLLLLLLLLLLLLLLLLLLLLLLL}
\tin{6}{miim\_ack}   \til{6}{LLLLLLLLLLLLLLLLLL....LLLLLLLLLLHHHHHHHHLLLLLL}
\tin{7}{miim\_rdata} \til{7}{UUUUUUUUUUUUUUUUUU....UUUUUUUUUUVVVVVVVVUUUUUU}
\sline{0.4}{4}{5.}
\sline{0.4}{12}{5.}
\sline{0.4}{22}{5.}
\sline{0.4}{30}{5.}
\sline{0.4}{38}{5.}
\end{timing}
\caption{Timing diagram of MII management read.}
\label{fig:timing-miim-rd}
\end{figure}

To perform a write to the management register, hold
\texttt{miim\_phyad}, \texttt{miim\_addr}, and \texttt{miim\_wdata}
valid and \texttt{miim\_we} and \texttt{miim\_req} high until
\texttt{miim\_ack} is asserted, at which point the write has been
performed.  This is illustrated in Figure~\ref{fig:timing-miim-wr}.

\begin{figure}
\begin{timing}[2s]{1in}
\tin{1}{usr\_clk}    \til{1}{LLLLHHHHLLLLHHHHLL....LLHHHHLLLLHHHHLLLLHHHHLL}
\tin{2}{miim\_phyad} \til{2}{UUUUVVVVVVVVVVVVVV....VVVVVVVVVVVVVVVVVVUUUUUU}
\tin{3}{miim\_addr}  \til{3}{UUUUVVVVVVVVVVVVVV....VVVVVVVVVVVVVVVVVVUUUUUU}
\tin{4}{miim\_req}   \til{4}{LLLLHHHHHHHHHHHHHH....HHHHHHHHHHHHHHHHHHLLLLLL}
\tin{5}{miim\_we}    \til{5}{LLLLHHHHHHHHHHHHHH....HHHHHHHHHHHHHHHHHHLLLLLL}
\tin{6}{miim\_ack}   \til{6}{LLLLLLLLLLLLLLLLLL....LLLLLLLLLLHHHHHHHHLLLLLL}
\tin{7}{miim\_wdata} \til{7}{UUUUVVVVVVVVVVVVVV....VVVVVVVVVVVVVVVVVVUUUUUU}
\sline{0.4}{4}{5.}
\sline{0.4}{12}{5.}
\sline{0.4}{22}{5.}
\sline{0.4}{30}{5.}
\sline{0.4}{38}{5.}
\end{timing}
\caption{Timing diagram of MII management write.}
\label{fig:timing-miim-wr}
\end{figure}

\subsection{UDP/IP Wrapper}

When the UDP/IP Wrapper is used, the ethertype port is removed and the
following configuration ports are added.

\vskip 12pt\noindent\begin{tabular}{p{0.85in}ccp{4.0in}}
\hline
Name & Width & Direction & Description \\
\hline

\hline src\_ip & 32 & I & IP address to use as the source of
trasmitted UDP/IP packets. \\

\hline src\_port & 16 & I & UDP port to use as the source of
transmitted UDP/IP packets. \\

\hline dst\_ip & 32 & I & IP address to send UDP/IP packets to. \\

\hline dst\_port & 16 & I & UDP port to send UDP/IP packets to. \\

\hline ttl & 8 & I & Value for the initial Time-To-Live field of
UDP/IP packets. \\

\hline
\end{tabular}\vskip 12pt

The transmission, reception, and management ports are all the same,
except instead of a four bit tx\_eof, there is a single bit tx\_eop,
because the UDP wrapper currently does not support sending sub 32-bit
aligned packets. 

\subsection{PHY}

The PHY ports are the standard GMII ports that should be tied directly
to pins connected to an ethernet PHY.

\begin{tabular}{p{0.85in}ccp{4.0in}}
\hline
Name & Width & Direction & Description \\
\hline

phy\_TXD      & 8 & O & \multirow{11}{4.0in}{These pins are the standard
  MII/GMII interface pins described numerous other places.} \\
phy\_TX\_EN   & 1 & O & \\
phy\_TX\_ER   & 1 & O & \\ 
phy\_TX\_CLK  & 1 & I & \\
phy\_GTX\_CLK & 1 & O & \\
phy\_RXD      & 8 & I & \\
phy\_RX\_DV   & 1 & I & \\
phy\_RX\_ER   & 1 & I & \\
phy\_RX\_CLK  & 1 & I & \\
phy\_MDC      & 1 & O & \\
phy\_MDIO     & 1 & IO & \\
\hline
\end{tabular}

\section{Architecture}

The MAC core is split up into three main components: tranmission,
reception, and reconciliation.  Figure\~ref{fig:mac-blocks} shows a
block diagram of how these components work together.  The transmission
blocks, \textbf{txfifo}, \textbf{tx\_engine}, and \textbf{crc\_gen},
are responsible for taking user data and sending it to the
reconciliation module, which sends it to the ethernet phy.  Likewise
the recption blocks, \textbf{rx\_usr\_if}, \textbf{rx\_pkt\_fifo},
\textbf{rxfifo}, \textbf{rx\_engine}, and \textbf{crc\_chk}, are
responsible for taking data from the reconciliation layer, throwing
away bad frames, and presenting it to the user.  The reconciliation
layer handles with the GMII interface to the phy and deals with
differences between 10/100Mbit operation and gigabit operation.

I have tried to make the verilog as readable and nicely commented as
possible, so hopefully the code is self documenting.  This section
just gives an overview of how things work.

\begin{figure}
\centering \includegraphics[width=\textwidth]{figs/mac-blocks.pdf}
\caption{Block diagram of the MAC core.}
\label{fig:mac-blocks}
\end{figure}

\subsection{Reconciliation}

The reconcilation modules provides a common interface to the rest of
the core regardless of the speed of the ethernet link.  This is done
by providing internal clocks to both the tx (\texttt{int\_tx\_clk})
and rx (\texttt{int\_rx\_clk}) modules which can be used to send data
to the reconciliation layer eight bits a clock.  

If \texttt{gmii} is high we are operating at gigabit speeds and the
phy sends and receives a full eight bits of data each 8ns.  For
transmission, \texttt{clk\_125} is used as both \texttt{int\_tx\_clk}
and sent as \texttt{phy\_GTXCLK}, synchronizing all tranmission to
that one 125 MHz clock.  Likewise, all reception is synchronized to
the 125 MHz \texttt{phy\_RXCLK}, which is sent as
\texttt{int\_rx\_clk}.

if \texttt{gmii} is low we are operating at 10 or 100 Mbits per second
and only four bits of the databus to the phy is used.  For
transmission, the phy expects four bits every cycle of the phy
supplied \texttt{phy\_TXCLK}; to maintain an eight bit interface with
the rest of the mac, \texttt{phy\_TXCLK} is divided in half and sent
as \texttt{int\_tx\_clk}.  Likewise, the phy sends four bits every
\texttt{phy\_RXCLK}, so this is divided in half and sent as
\texttt{int\_rx\_clk}.

\subsection{Transmission}

For transmission, \textbf{tx\_engine} runs a simple state machine
which controls what is sent to the reconciliation layer, pulling data
from the configuration ports for the header, \textbf{txfifo}, a 36 bit
wide asynchronous fifo, for the payload, and \textbf{crc\_gen} for the
final CRC checksum.  \textbf{txfifo} is 36 bits wide to hold both the
data and the corresponding end of frame bits.

If \textbf{txfifo} runs out of data, \textbf{tx\_engine} raises
\texttt{phy\_TXER} to drop the packet being sent and continues to
drain \textbf{txfifo} until an end-of-frame is indicated.  It then
moves on to the next packet in \textbf{txfifo}.

A frame being sent will also be dropped if the frame ends too early or
too late.  A frame must be 46 bytes long to be sent. The
\texttt{jumobframes} input changes when \textbf{tx\_engine} thinks a
frame as grown too large and should be dropped.  If
\texttt{jumboframes} is low, the frame will be dropped if larger than
1500 bytes, otherwise it will be dropped if larger than 9000. Note
that the phy may or may not accept frames larger than the standard
1500 bytes, but the MAC has no way of knowing this.

Once a packet has successfully been sent, \textbf{tx\_engine} wait for
an inter frame gap, set by the \texttt{IFG} parameter to the
\textbf{tx\_engine} module, which defaults to 12 cycles and begin
sending the next frame if it is available.

\subsection{Reception}

Reception is controlled by both \textbf{rx\_engine} and
\textbf{rx\_usr\_if}.  \textbf{rx\_engine} receives data from the phy
and passes all the data unaltered, except for the preamble, to the
\textbf{rxfifo}, once a full packet is received it writes a '1' to the
single bit asynchronous fifo \textbf{rx\_pkt\_fifo} to indicate to
\textbf{rx\_usr\_if} that the data in \textbf{rxfifo} is ready to send
to the user.  If an error is detected during packet reception, either
because \texttt{phy\_RXER} was asserted, the frame was too short or
too long, or the destination MAC address didn't match and
\texttt{promiscuous} was not asserted, then a '0' is written to the
\textbf{rx\_pkt\_fifo} to indicate to \textbf{rx\_usr\_if} that a bad
packet is in \textbf{rxfifo}.  \textbf{rx\_usr\_if} will then drain
the bad packet from \textbf{rxfifo} without ever presenting it to the
user.

\subsection{Management}

Implementation of the management interface is just a straightforward
state machine which performs the requested read and write operations
using the MII managment protocol.  See section 22.2.2 of IEEE
802.3-2008, or any phy's datasheet, for a description of the protocol.

\subsection{Clocking}

There are six different clock domains within the MAC core, delineated
by the dotted lines in Figure~\ref{fig:mac-blocks}. 

\begin{itemize}
\item{\texttt{usr\_clk}} This is the only clock that the user of the MAC needs
  to be aware of, as all user signals into the MAC are synchronized to
  this clock.  It can be any speed as long as it is fast enough as
  detailed in Section~\ref{sec:tx-iface}.

\item{\texttt{int\_tx\_clk}} This clock is used by all of the transmission
  modules and is generated by the reconciliation layer depending on
  the speed of the ethernet link. It is either \texttt{clk\_125} or
  half of \texttt{phy\_TXCLK}.

\item{\texttt{int\_rx\_clk}} The reception version of \texttt{int\_tx\_clk},
  also generated by the reconiliation layer.  It is either
  \texttt{phy\_RXCLK} or half of \texttt{phy\_RXCLK}.

\item{\texttt{phy\_TXCLK/phy\_GTXCLK}} This is used to clock data out of the
  FPGA to the phy.  \texttt{phy\_TXCLK} is supplied by the phy and
  \texttt{phy\_GTXCLK} is \texttt{clk\_125}.

\item{\texttt{phy\_RXCLK}} This is used to clock data into the FPGA from the
  phy and is supplied by the phy.

\item{phy\_MDC} This is used to clock data in and out of the phy's
  management port.  It is generated from \texttt{usr\_clk} based on
  the \texttt{USR\_CLK\_PERIOD} parameter to the MAC core.  It should
  have a period greater than 400ns as per section 22.2.2.11 of IEEE
  802.3-2008.
\end{itemize}

\end{document}
